\documentclass[a4j]{jarticle}
\usepackage{graphicx}
\usepackage{jcfd39}
\usepackage{url}
%
%\usepackage{ulem} %proofreading e.g. \uuline{ \uwave{  \sout{demo}}}
%
%\makeatletter  % Change the style of citation and biblabal - II
%\def\@cite#1{$^{\mbox{\scriptsize{(#1})}}$}
%\makeatother

\newcommand{\ctext}[1]{\raise0.2ex\hbox{\textcircled{\scriptsize{#1}}}}


%%\def\papernum{講演番号}                %講演番号を記して下さい

\title{2次元直交格子を用いたshock-fitting法の基礎的研究}   %和文タイトルを記して下さい
\etitle{Fundamental Study on the Shock-Fitting Method Using 2D Cartesian grid}   %英文タイトルを記して下さい
                                    %和文著者名を記して下さい
\author{\begin{tabular}{cl}
$\bigcirc$ & 西尾朋人,東大, 
             東京都文京区本郷7-3-1, 
             E-mail:nishio-tomohito048@g.ecc.u-tokyo.ac.jp \\
           & 今村太郎,東大,
             東京都文京区本郷7-3-1, 
             E-mail:imamura@g.ecc.u-tokyo.ac.jp
\end{tabular} }

                                    %英文著者名を記して下さい
\eauthor{\begin{tabular}{l}
    Tomohito NISHIO, 
    The University of Tokyo, 
    7-3-1 Hongo, Bunkyo-ku, Tokyo, Japan \\
    Taro IMAMURA, 
    The University of Tokyo, 
    7-3-1 Hongo, Bunkyo-ku, Tokyo, Japan
\end{tabular} }

                                    %アブストラクトを記して下さい
\abstract{%
%
    Abstract must be 100 -- 150 words using 9pt Times font.
    This is a simple example of how to prepare the paper for CFD39.
    The headings should appear as above.
    The instruction is written in the main body.
    Abstract must be 100 -- 150 words using 9pt Times font.
    This is a simple example of how to prepare the paper for CFD39.
    The headings should appear as above. The instruction is written in the main body. 
%
}
\begin{document}
\baselineskip=1zw
\maketitle
                                    %ここから本文です

\section{序論} \label{sec:intro}

\section{手法} \label{sec:method}
\mbox{}\\[-5.0ex]
\subsection{Shock-Capturing法 Euler方程式ソルバー} \label{subsec:sc}
\mbox{}\\[-3.0ex]

圧縮性Euler方程式を直交格子有限体積法(FVM)で解く.
セル内物理量の再構築にはMUSCL法を用いる.
物理量の勾配計算はWLSQ(G)法により行い, 制限関数としてVan Albada リミッターを用いる.
近似RiemannソルバーとしてAUSM族 SLAUスキームを使用する.
時間積分は陽解放の3次精度のTVD Runge-Kutta法で行う.
CFL条件は各セルの幅$\Delta x$, $\Delta y$, 流体速度$u_{i,j}$, $v_{i,j}$, 音速$a_{i, j}$を用いて, 式~(\ref{eq:CFLcon})で定める。
\begin{equation}
    \Delta t =
    \frac{\mathrm{CFL}}{
        \max\left(
            \frac{|u_{i,j}|}{\Delta x}
            + \frac{|v_{i,j}|}{\Delta y}
            + a_{i,j}
              \sqrt{
                \frac{1}{\Delta x^2}
                + \frac{1}{\Delta y^2}
              }
        \right)
    }
    \label{eq:CFLcon}
\end{equation}
直交格子に沿わない物体境界での境界条件には, 埋め込み境界法(Immersed Boundary Method, IB法)を用いる.

\subsection{不連続領域の判定・抽出} \label{subsec:discon}
\mbox{}\\[-7.0ex]
\subsubsection{圧縮性の顕著なセルの特定} \label{subsubsec:comp}
\mbox{}\\[-1.0ex]

セルごとの圧縮性の定量的な評価法として, 速度場の発散を用いる.
SC法CFDの計算結果から, 各セルで速度の発散を計算し, その2乗平均平方根(RMS)を算出する.
速度の発散がこのRMSより大きいセルを, 圧縮性が顕著なセルとし, 不連続領域と特定する.

\subsubsection{Connected Component Labeling}\label{subsubsec:CCL}
\mbox{}\\[-1.0ex]

[\ref{subsubsec:comp}]で評価した圧縮性の顕著なセルは,数値的な速度発散に基づく定義であるため,物理的に意味のない領域(ノイズや離小島)を含む場合がある.
これらの不要な領域を自動的に除去し, 物理的に意味のある高圧縮性領域のみを抽出するために, Connected Component Labeling(CCL)によるクラスタリング処理を適用する.
[\ref{subsubsec:comp}]で評価された圧縮性の顕著なセルについて, 同様のセルを縦横斜め8近傍に持つ場合に, それらを同一グループとしてクラスタリングしていく.
得られた各クラスターごとにそのサイズ(属するセル数)を計算し,計算領域の全セル数に対して一定割合未満の小さい連結グループをノイズとして除去対象とする.
こうして,対象となる比較的大きなクラスターのみを残し,物理的に意味のある領域を選別可能とした.
この処理にはPythonの数値処理ライブラリ \texttt{NumPy} および \texttt{SciPy} の \texttt{ndimage.label} 関数を用いる.

\subsubsection{不連続領域のSkeleton化} \label{subsubsec:skeleton}
\mbox{}\\[-1.0ex]

\texttt{Python} の \texttt{skimage.morphology skeletonize} 関数を用いて, [\ref{subsubsec:CCL}]で得たクラスターの主要骨格をセル1ピクセル幅で取得する.
この結果を\texttt{Python} の \texttt{networkx}を用いて処理する.
本研究では干渉のない一つの衝撃波を扱うため, 主要骨格セルの中の最長経路を特定する.
\texttt{networkx}は分岐や経路処理に強いため,
今後衝撃波干渉や分岐などを扱うにあたっても, この手法は応用性が高いと考えた.

\subsubsection{衝撃波面のSpline近似} \label{subsubsec:spline}
\mbox{}\\[-1.0ex]

[\ref{subsubsec:skeleton}]で得た不連続領域の最長骨格に対し, 
\texttt{SciPy.interpolate} の \texttt{splprep, splev} 関数を用いてspline補完を行う.
ここで, ユーザー任意のサンプリング点数のshock-pointsを任意の滑らかさで用意できる.

\subsection{Shock-Fitting法}
\mbox{}\\[-3.0ex]

[\ref{subsec:sc}],[\ref{subsec:discon}]で, SC法CFDの計算結果と初期衝撃波位置, shock-pointsが用意される.
ここで, 直交格子のセルは各1つの物理量を持つのに対し, shock-pointsは衝撃波直前と直後の2つの値を持つ.
Shock-Fitiing法(SF法)では, はじめに不連続領域を除く衝撃波上流と下流で独立してSC法CFDを行い, タイムステップを$\Delta t$進める.
次に, その情報をshock-pointsに外挿し, Rankine-Hugoniot条件を課すことで, shock-pointsでの上流・下流の2つの物理量とshock-pointsの速度が
決定される. これにより, 不連続領域でもタイムステップを$\Delta t$進めることができる. この情報をフィードバックし, 再び不連続領域を除く衝撃波上流と下流で独立してSC法CFDを実行する.
このプロセスの繰り返しにより, 直交格子のセルとshock-pointsでの物理量, shock-pointsの位置が時間発展する. 

\subsubsection{衝撃波面と交差するBlanked cellsのラベリング} \label{subsubsec:exclude}
\mbox{}\\[-1.0ex]


\subsubsection{Adjoining Boundary cells, Surrogate Boundary cellsのラベリング} \label{subsubsec:boundary}
\mbox{}\\[-1.0ex]


\subsubsection{上流セル・下流セルの識別} \label{subsubsec:udflow}
\mbox{}\\[-1.0ex]


\subsubsection{不要なShock Nodesの除去} \label{subsubsec:remove}
\mbox{}\\[-1.0ex]


\subsubsection{SC法ソルバーによる計算領域でのCFDの実行} \label{subsubsec:sc}
\mbox{}\\[-1.0ex]


\subsubsection{衝撃波法線ベクトルの計算} \label{subsubsec:normal}
\mbox{}\\[-1.0ex]


\subsubsection{Surrogate Boundary cellsからShock Nodesへの値の外挿} \label{subsubsec:extrapolate}
\mbox{}\\[-1.0ex]


\subsubsection{Shock NodesでのRankine-Hugoniot条件の適用} \label{subsubsec:rankine}
\mbox{}\\[-1.0ex]


\subsubsection{時間積分によるShock Nodesの進行} \label{subsubsec:prop}
\mbox{}\\[-1.0ex]


\subsubsection{Shock NodesからAdjoining Boundary, Blanked cellsへの値の外挿} \label{subsubsec:adjoining}
\mbox{}\\[-1.0ex]


\section{結果}
\subsection{1次元非定常垂直衝撃波}

\subsection{2次元定常斜め衝撃波}

\subsection{2次元定常角柱離脱衝撃波}

\subsection{2次元定常円柱離脱衝撃波}

\section{結論}

\section*{参考文献}

\begin{enumerate}
    \item 荒川,谷口, ``論文の書式について,'' 第17回数値流体力学講演論文集, 1 (2003), pp. 1-1.
    \item Arakawa, C. and Taniguchi, N., ``How to prepare the paper,'' Proc. 17th CFD Symp., 1 (2003), pp. 1-1.
\end{enumerate}

\end{document}
